\documentclass[11pt]{amsart}

\usepackage{amssymb}
\usepackage{latexsym}
\usepackage{amsmath}

\begin{document}
\noindent
I am interested in a test that determines that an out of time k-step forecast of an ARX model is a "good" forecast (i.e., residuals are as close to zero as possible).\\

\noindent
Let's say we have a $ARX(1)$ model developed on $m$ observations and it has $n$ co-variates.

\begin{equation}
Y_t=\alpha Y_{t-1}+\sum_{i=1}^n \beta_i X_{it} + \epsilon_t
\end{equation}\label{eqn1}


Reference to Eqn1 goes here \eqref{eqn1}.
\noindent
The error terms are i.i.d. normal $N(0,\sigma)$ I go on and build a $k$ step forecast over a data period that follows the original $m$ observations (i.e., out of time forecast). I attempt to evaluate residuals recursively.

The first forecasted point is
$$ \hat{Y}_1=\alpha Y_0 + \sum_{i=1}^n \beta_i X_{i1} $$
the corresponding residual will be
$$r_1 =  Y_1-\hat{Y_1} = \epsilon_1$$
which follows that $r_1 \thicksim N(0,\sigma)$. Moving on to the second forecasted point (Note this uses the forecast from the first step.)

$$ \hat{Y}_2=\alpha \hat{Y}_1 + \sum_{i=1}^n \beta_i X_{i1} $$

\noindent
Since $\hat{Y}_1=Y_1 - \epsilon_1$

$$r_2= Y_2-\hat{Y}_2 =Y2 - \alpha Y_1 - \sum_{i=1}^n \beta_i X_{i2}+\alpha \epsilon_1 =\epsilon_2+\alpha \epsilon_1$$

\noindent
The distribution of $$r_2\thicksim N(0,\sigma\sqrt(1+\alpha^2)) $$ repeating the process we get that 

$$r_j=\sum_{k=0}^{j-1} \epsilon_{j-k} \alpha^k$$

\noindent
and the distribution of $$r_j \thicksim N\left((0,\sigma^2 \sqrt{\sum_{k=0}^{j-1}\alpha^k} \right) $$

\noindent
Now, I am interested in answering a question if the k-step forecast is good and the model I have does not need to be redeveloped. The statistic I am think that would work for a potential test is sum of normalize residual squares.

$$ Stat= \sum_{i=1}^k \frac{r_i^2}{\sigma^2\sum_{l=0}^{i-1}\alpha^l} $$

\noindent
the above Stat will be $\chi^2(k)$. I am thinking I should be able to use this Stat to test a null hypothesis $r_i=0 \ \forall i=1...k$ v.s. alternative $r_i\neq 0 \ \forall i=1...k$. Leads to questions.\\

\noindent
1) Is this a worthwhile test to evaluate performance of a model?\\ \\

\noindent
2) Have you ever seen anyone do something like this? I am basing this off Chow test, Chow, Gregory C. (1960). "Tests of Equality Between Sets of Coefficients in Two Linear Regressions". Econometrica . The difficulty with applying the test in that paper is that its for OLS and the residuals are all i.i.d. normal, whereas in time series model the k-step forecast residuals have changing distribution. \\ \\


\end{document}
